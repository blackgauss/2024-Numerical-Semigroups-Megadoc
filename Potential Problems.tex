\documentclass{article}
\usepackage{graphicx} % Required for inserting images
\usepackage{amsmath}
\usepackage{amssymb}
\usepackage{amsthm}
\usepackage{tikz}
\usepackage{ytableau}
\usepackage{tabularx}

\theoremstyle{definition}
\newtheorem{thm}{Theorem}[section]
\theoremstyle{definition}
\newtheorem{defn}[thm]{Definition}
\theoremstyle{definition}
\newtheorem{ex}[thm]{Example}
\newtheorem{prop}[thm]{Proposition}

%\title{}
%\date{2024}
\begin{document}

I am listing potential problems to work on and interesting questions raised during the meetings. Feel free to add to the list.

\section{Distribution of size of partitions with given hook-lengths}
\begin{itemize}
    \item 
    \begin{enumerate}
    \item Given a numerical semigroup $S$, what can you say about the distribution of $n(T)$ as $T$ varies through numerical sets with $A(T)=S$.
    \item Do part 1) when $S$ is in the following families:
    \begin{enumerate}
        \item[a] $S(n)=\{0,n+1\rightarrow\}$. (M$\&$M paper asymptotically computes $P(S)$ for this family)
        \item[b] $S(n)=\{0,5,10,\dots,5n,5n+2\rightarrow\}$. (2019 REU paper 2, computes $P(S)$ for this and next families)
        \item[c] $S(n)=\{0,5,10,\dots,5n,5n+4\rightarrow\}$.
        \item[d] $S(n)=\{0,6,12,\dots,6n\rightarrow\}$.
    \end{enumerate}
    \item Given a numerical semigroup $S$, what can you say about the distribution of $n(T)$ as $T$ varies through the numerical sets with $A(T)=S$ for which $T\setminus S$ is \emph{self-dual} (This is not the same as the partition of $T$ being symmetric).
\end{enumerate}
    \item We say a numerical set $T$ has no small atoms if $A(T)=\{0,F(T)+1\rightarrow\}$. Denote the number of partitions of $n$, by $p(n)$. Show that
    \[\lim_{n\to\infty} \frac{\#\{T\mid n(T)=n, T \text{ has no small atoms}\}}{p(n)}=0.\]
    \item
    \begin{enumerate}
        \item Given a numerical semigroup $S$, what is $\min\{n(T)\mid A(T)=S\}$? (what about $\max$?)
        \item Under what condition on $S$ can you conclude that $\min\{n(T)\mid A(T)=S\}=n(S)$?
        \item Do "most" numerical semigroups $S$ satisfy $\min\{n(T)\mid A(T)=S\}=n(S)$?
        \item 
        \[f(n)=\min\{|\lambda(T)|\mid A(T)=S, \lambda(S)=n\}.\]
        Describe the growth of $f$.\\
        Alternately, describe a function $f$ (and try to make it as big as possible), such that:
        For sufficiently large $n$, if $S$ is a numerical semigroup with $n(S)=n$ and $A(T)=S$, then $n(T)\geq f(n)$.\\
        We have a family that shows we must have $f(n)\leq 0.8 n$.\\
        We (maybe) have a family that shows $f(n)\leq \epsilon n$ for any $\epsilon>0$.\\
        Would $f(n)=n^\alpha$ work for some constant $0<\alpha<1$?
    \end{enumerate}
\end{itemize}


\section{Numerical semigroups of size $n$}

\begin{itemize}
    \item Consider the set of all numerical semigroups for which $n(S)=n$.
\begin{enumerate}
    \item Among them what is the largest/smallest Frobenius number?
    \item Among them what is the largest/smallest genus?
    \item Among them what is the largest/smallest Durfee square?
    \item Do most of them have Frobenius number close to some value?
    \item Do most of them have genus close to some value?
    \item Do most of them have multiplicity number close to some value?
\end{enumerate}




\end{itemize}









\section{Counting numerical semigroups of size $n$}
Let $p(n)$ be the number of partitions of $n$. So $p(n)=\#\{T\mid n(T)=n\}$. It is known that
\[p(n)\sim \frac{1}{4\sqrt{3} n}\exp\Big(\pi\sqrt{\frac{2}{3}}\sqrt{n}\Big).\]
Denote $p_{NS}(n)=\#\{S\mid n(S)=n\}$.

\begin{itemize}
    \item Show that
    \[\lim_{n\to\infty} \frac{p_{NS}(n)}{p(n)}=0.\]
    \item Does the following limit exist
    \[\lim_{n\to\infty} \frac{\log(n\times p_{NS}(n))}{\sqrt{n}}?\]
    \item What if you count the number of symmetric numerical semigroups whose partitions have size $n$?
\end{itemize}


\section{Void poset and Partition of NS}
\begin{itemize}
    \item Which finite posets $(X,\preccurlyeq)$ occur as a void poset of a numerical semigroup?
    \begin{enumerate}
        \item It needs to be self dual, i.e., there is a bijection $\phi:X\to X$, such that $x\preccurlyeq y$ if and only if $\phi(y)\preccurlyeq \phi(x)$.
        \item The duality map $\phi: X\to X$, has exactly one fixed point if $|X|$ is odd, and no fixed points if $|X|$ is even.
        \item For $x,y\in X$, if $x\preccurlyeq y$, then the number of elements of $X$ directly above $x$ is at least as many as directly above $y$.
        \item What else???
        \item If there is at least one NS whose Void poset is $(X,\preccurlyeq)$, then are there infinitely many NS whose Void poset is $(X,\preccurlyeq)$?
    \end{enumerate}
    \item What properties of a numerical semigroup can be deduced by looking at its void poset?
    \begin{enumerate}
        \item $S$ is symmetric iff $M(S)=\emptyset$.
        \item $S$ is pseudo-symmetric iff $|M(S)|=1$.
        \item $S$ is almost-symmetric iff $M(S)$ has no relations.
        \item What else???
    \end{enumerate}

    \item What properties of a numerical semigroup $S$ can be seen from $\lambda(S)$.
\begin{enumerate}
    \item $F(S)$ is the largest hook length.
    \item $g(S)$ is the number of rows
    \item The gaps of $S$ are the hook lengths of $\lambda(S)$.
    \item Pseudo-Frobenius numbers of $S$ the hook lengths that occur exactly once (proven this summer).
    \item Void of $S$ consists of hooklengths that don't occur the the top row (proven this summer).
    \item The Void Poset can be seen from $\lambda(S)$.
    \item Can $GPF(S)$ be seen from $\lambda(S)$?
    \item A numerical semigroup has type $1$ if and only if $\lambda(S)$ is self conjugate.
    If $t(S)=2$, is there something interesting in $\lambda(S)$?
    \item Given $\lambda(S)$, we can use it to construct the Gap Poset. We have shown that the number of relations in the Gap poset (including $x\preccurlyeq x$) is the size of $\lambda(S)$.\\
    Can we use the Gap Poset of $S$ to construct $\lambda(S)$?\\
    If $T$ is a numerical set, we have a partition $\lambda(T)$. Is there a generalization of Gap Poset which maintains the correspondence?\\
    If $S$ is a numerical semigroup, and $I\subseteq H(S)$ is an order ideal of the gap poset. Can the size of $\lambda(S\cup I)$ be seen from the Gap poset?
    \item If $S$ has embedding dimension $2$, then $S$ is symmetric, so $M(S)=\emptyset$ and $\lambda(S)$ is self conjugate.\\
    If $S$ has embedding dimension $3$, it appears from some examples that $\lambda(S)$ "appears to be" self dual (of course it is not actually self dual). Can you turn this into a precise statement? The Void poset also appears to be "nice and orderly", can this be made precise?
    \item Is there a relation between the height of void poset of $S$ and how much $\lambda(S)$ "appears" to be self conjugate?
\end{enumerate}
\end{itemize}



\section{Effective weight}

\begin{itemize}
    \item Pfluger conjecture: given any numerical semigroup $S$, it satisfies:
    \[ewt(S)\leq \frac{(g(S)+1)^2}{8}\]
    \begin{enumerate}
        \item Prove Pfluger conjecture for numerical semigroups of depth $2$.
        \item Prove Pfluger conjecture for numerical semigroups of depth $3$.
        \item Prove Pfluger conjecture for numerical semigroups of multiplicity at most $N$ (choose $N$, make it as big as you can).
        \item Find other intermediate steps
        \item Prove Pfluger conjecture in general.
    \end{enumerate}
\end{itemize}

\section{NS tree}

If a nuerical semigroup $S'$, is a child of $S$ in the semigroup tree. Then How are properties of $S'$ related to those of $S$?
\begin{enumerate}
    \item Compare their void posets.
    \item Compare $P(S)$.
    \item What else?
\end{enumerate}


\end{document}