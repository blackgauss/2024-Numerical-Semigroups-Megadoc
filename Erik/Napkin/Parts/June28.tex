\subsection{Partitions and Numerical Sets}

This material is background but its worth recalling it here. There exists a bijection
\[\varphi: \textsf{Numerical Sets} \to \textsf{Partitions}\]

constructed by forming walks where each step is decided by an elements membership in a numerical set. A formal construction could look something like this:
\[e_T: \N_{\le \frob(T)} \to \{\rightarrow, \uparrow\}^{\frob(T) + 1}\]

\[e_T(x): \begin{cases} \rightarrow & x \in T \\ \uparrow & x \not\in T\end{cases}\]

but the description is up to personal preference.

The image of \(e_T\) describes a walk which is equivalent to the profile of a partition\footnote{The walk is finite because after \(\frob(T)\) every step is a right step. We cut the walk off at this point}. The correspondence \(T \sim \im(e_T)\) gives a bijection between numerical sets and partitions. The details can be found in \cite{Constantin2017}. The following is a worked example for constructing a partition from a numerical set.

\begin{examplebox}
    Let \(T\) be the numerical set defined by \(\N_0 \setminus T = \{1, 2, 4\}\) so that \(\frob(T) = 4\). We construct the walk by finding the the steps associated to each element less than or equal to \(\frob(T) = 4\).

    \begin{center}
        \begin{tabular}{c|c}
        Element & Step \\
        \hline
        \(0\) & \(\rightarrow\)  \\
        \(1\) & \(\uparrow\) \\
        \(2\) & \(\uparrow\) \\
        \(3\) & \(\rightarrow\) \\
        \(4\) & \(\uparrow\) \\
        \end{tabular}
    \end{center}

    Starting at \(0\) and taking each step gives the following walk:
        \begin{center}
    \begin{center}
    \begin{tikzpicture}[scale=1, >=Stealth]

    % Draw the grid
    \draw[very thin, gray] (0,0) grid (5,3);

    % Draw the axes
    \draw[->] (0,0) -- (5.5,0) node[right] {x};
    \draw[->] (0,0) -- (0,3.5) node[above] {y};

    % Draw the path
    \draw[->, thick] (0,0) -- (1,0) node[midway, below] {\(0\)};
    \draw[->, thick] (1,0) -- (1,1) node[midway, right] {\(1\)};
    \draw[->, thick] (1,1) -- (1,2) node[midway, right] {\(2\)};
    \draw[->, thick] (1,2) -- (2,2) node[midway, below] {\(3\)};
    \draw[->, thick] (2,2) -- (2,3) node[midway, right] {\(4\)};

    % Remove the tick labels on x and y axes
    \foreach \x in {0, 1, 2, 3, 4, 5}
        \draw (\x, 0) -- (\x, -0.1);

    \foreach \y in {0, 1, 2, 3}
        \draw (0, \y) -- (-0.1, \y);

    \end{tikzpicture}
    \end{center}
    \end{center}

    This region to the left of the walk is the partition \(2 + 1 + 1\) which corresponds to the following Ferrers diagram: {\tiny \ydiagram[*(white) ]{2, 1, 1}}. From now on we will omit the arrow heads on the steps of the walk since arrow heads add no value.
\end{examplebox}

The hook lengths of the partition encode information about the walk and therefore the numerical set associated to the walk. It is useful to state some computational tricks.

\subsection{Computational Tricks Using Hooks}

Given a labeled walk, which amounts to a function \(e_T\), we can find the hook lengths by noting each box uniquely determines a right step and an up step.

\begin{center}
    \begin{minipage}{0.5\textwidth}
        \centering
        \begin{tikzpicture}[scale=1, >=Stealth]

        % Draw the grid
        \draw[very thin, gray] (0,0) grid (8,4);

        % Draw the axes
        \draw[->] (0,0) -- (8.5,0) node[right] {x};
        \draw[->] (0,0) -- (0,4.5) node[above] {y};

        % Draw the path
        \draw[thick] (0,0) -- (1,0) node[midway, below] {\(0\)};
        \draw[thick] (1,0) -- (1,1) node[midway, right] {\(1\)};
        \draw[thick] (1,1) -- (2,1) node[midway, below] {\(2\)};
        \draw[thick] (2,1) -- (3,1) node[midway, below] {\(3\)};
        \draw[thick] (3,1) -- (3,2) node[midway, right] {\(4\)};
        \draw[thick] (3,2) -- (3,3) node[midway, right] {\(5\)};
        \draw[thick] (3,3) -- (4,3) node[midway, below] {\(6\)};
        \draw[thick] (4,3) -- (4,4) node[midway, right] {\(7\)};

        % Remove the tick labels on x and y axes
        \foreach \x in {0, 1, 2, 3, 4, 5, 6, 7, 8}
            \draw (\x, 0) -- (\x, -0.1);

        \foreach \y in {0, 1, 2, 3, 4}
            \draw (0, \y) -- (-0.1, \y);

        % Draw the hook in the box with center (1.5, 2.5)
        \fill[blue] (1.5, 2.5) circle (0.2); % Filled blue circle in the middle of the specified box
        \draw[blue, thick] (1.5, 2.5) -- (2.5, 2.5); % Blue line extending to the center of the box to the right
        \draw[blue, thick] (1.5, 2.5) -- (1.5, 1.5); % Blue line going down to the center of the box below

        \end{tikzpicture}
    \end{minipage}%
    \begin{minipage}{0.5\textwidth}
        \centering
        {\Large \ytableausetup{centertableaux}
        \begin{ytableau}
            7 & 5 & 4 & 1 \\
            6 & 3 & 2 \\
            4 & 2 & 1 \\
            1 \\
        \end{ytableau}}
    \end{minipage}
\end{center}

The length of the hook can be be found be taking the index of the up step (at the right of the arm) the index of the right step (at the bottom of the leg). In the diagram above the blue hook length is \(5 - 2 = 3\). Recall that the length of the hook is also the number of boxes the hook crosses. However, thinking in terms of the step indices helps us understand what each box can tell us about the numerical set associated to the walk.

\begin{itemize}
    \item The leftmost column contains all the indices which are up steps. Therefore the leftmost column has exactly the gaps of the numerical set as its hook lengths.
    \item The topmost row contains hook lengths of the form \(\frob(T) - x\) for each \(x \in T\). The void \(M(T)\) contains the elements in the leftmost column that do not appear in the topmost column. We can also find all small elements of \(T\) by taking subtracting the top row hook lengths from \(F(T)\).
\end{itemize}

\begin{examplebox}
    \begin{center}
    \begin{minipage}{0.5\textwidth}
        \centering
        {\Large \ytableausetup{centertableaux}
        \begin{ytableau}
            7 & 5 & 4 & 1 \\
            6 & 3 & 2 \\
            4 & 2 & 1 \\
            1 \\
        \end{ytableau}}
        \\[1em]
        \textbf{Hook lengths}
    \end{minipage}%
    \begin{minipage}{0.5\textwidth}
        \centering
        {\Large \ytableausetup{centertableaux}
        \begin{ytableau}
            0 & 2 & 3 & 6 \\
            1 & 4 & 5 \\
            3 & 5 & 6 \\
            6 \\
        \end{ytableau}}
        \\[1em]
        \textbf{\(\frob(T)\) minus hook lengths}
    \end{minipage}
    \end{center}
\end{examplebox}





\bibliography{Erik/bibtex/references}