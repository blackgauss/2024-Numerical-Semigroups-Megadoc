\section{Occurrences of Hooks and Gapsets}

\begin{definition}
Let \( S \) be a numerical semigroup. The \textbf{gaps} of \( S \) form a partially ordered set under the relation \( \ge_S \), defined by \( y \ge_S x \) whenever \( y - x \in S \). This partially ordered set is referred to as the \textbf{gap poset}.
\end{definition}

\begin{proposition}
    The pseudo-Frobenius numbers are the maximal elements in the gap poset.
\end{proposition}

\begin{proof}
    Suppose \( p \) is a maximal element in the gap poset. This means there does not exist a gap \( h \) such that \( h - p \in S \). In other words, \( p + S \subseteq S \) and \( p \) is a pseudo-Frobenius number.

    Conversely, if \( p \) is a pseudo-Frobenius number, then \( p + S \subseteq S \). Suppose, for the sake of contradiction, that there exists a gap \( h \) such that \( h - p \in S \). Then \( h = p + s \) for some \( s \in S \), which would imply that \( h \) is in \( S \), contradicting the fact that \( h \) is a gap.
\end{proof}

We say a row corresponds to a gap \(f\) if \(f\) appears in the first column of the given row. 

\begin{proposition}
A gap \( h \) appears in the row corresponding to a gap \( f \) if and only if \( f \ge_S h \).
\end{proposition}

\begin{proof}
Suppose \( h \) appears in the row of a gap \( f \). Then by definition, \( h = f - s \) for some \( s \in S \). Therefore, \( f - h = s \), implying \( f \ge_S h \). 

Conversely, if \( f \ge_S h \), then \( f - h = s \) for some \( s \in S \). Hence, \( h = f - s \) must appear as a hook in the row of \( f \).
\end{proof}

\begin{corollary}
    The pseudo-frobenius numbers are the hooks which only appear once in the partition of \(S\).
\end{corollary}

Similarly, the number of times a hook appears counts how many gaps are above it in the gap poset.

\begin{corollarly}
The number of relations in the gap poset is equal to the number of boxes in the partition, which is the size of the partition.
\end{corollarly}

\iffalse
\begin{proof}
    Each relation in the gap poset corresponds to a box in the Ferrers diagram of the partition. Since every box represents a unique relation where a hook \( h \) can be determined from a gap \( f \) and a semigroup element \( s \), the total number of such relations is precisely the number of boxes in the partition.
\end{proof}
\fi