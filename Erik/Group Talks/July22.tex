\documentclass[11pt]{article}
\usepackage{fullpage}
\usepackage[top=2cm, bottom=4.5cm, left=2.5cm, right=2.5cm]{geometry}
\usepackage{amsmath,amsthm,amsfonts,amssymb,amscd}
\usepackage{lastpage}
\usepackage{enumerate}
\usepackage{fancyhdr}
\usepackage{mathrsfs}
\usepackage{xcolor}
\usepackage{graphicx}
\usepackage{listings}
\usepackage{hyperref}
\usepackage[capitalize,nameinlink,noabbrev]{cleveref}

\usepackage{tikz-cd}
\usepackage{tikz}
\usepackage{pgfplots}

\usepackage{todonotes}
    \pgfplotsset{
        compat=1.12,
    }
\usepackage{physics}
\usepackage{draftwatermark}
\usepackage{tikz}
\usetikzlibrary{arrows.meta}
\usetikzlibrary{positioning}
\usetikzlibrary{trees}
% \usepackage{emoji}

\usepackage{pdflscape}
\usepackage{ytableau}



\usepackage[breakable,skins,most]{tcolorbox}

% tcolorboxes
\usepackage{Erik/Styles/colorboxes}
 
\renewcommand\lstlistingname{Algorithm}
\renewcommand\lstlistlistingname{Algorithms}
\def\lstlistingautorefname{Alg.}

\lstdefinestyle{Python}{
    language        = Python,
    frame           = lines, 
    basicstyle      = \footnotesize,
    keywordstyle    = \color{blue},
    stringstyle     = \color{green},
    commentstyle    = \color{red}\ttfamily
}

\setlength{\parindent}{0.0in}
\setlength{\parskip}{0.05in}

% Edit these as appropriate
\newcommand\course{UCI 2024}
\newcommand\hwnumber{}                 
\newcommand\Name{Erik Imathiu-Jones}

% Macros
\usepackage{Erik/Styles/macros}

\SetWatermarkText{}%\emoji{rose}} %\emoji{smiling-face-with-horns}}
\SetWatermarkScale{1}
\SetWatermarkAngle{0}

% Bibliography setup
\usepackage{natbib} % For more flexible citations
\bibliographystyle{plainnat} % Specify bibliography style
\usepackage{url} % To include URLs in bibliography


\pagestyle{fancyplain}
\headheight 35pt
\lhead{\Name}
\chead{\textbf{\Large July 22 Team Talk \hwnumber}}
\rhead{\course}
\lfoot{}
\cfoot{}
\rfoot{\small\thepage}
\headsep 1.5em

\title{Partition Land: A Collection of Ideas from Summer 2024}
\author{Erik Imathiu-Jones}
\setcounter{tocdepth}{3}
\setcounter{secnumdepth}{3}

\begin{document}

%\maketitle

\newpage
\section{Outline}[2 mins]
\begin{enumerate}
    \item Background (depth) [1 min]
    \item Depth 2 - Partitions [4 min]
    \item Why its easy [3 mins]
    \item Counting depth 2 [2 mins]
    \item Plueger's in the depth 2 [4 mins]
    \item Depth 3 an onward [3 mins]
\end{enumerate}

\section{So... what is depth again?}
\begin{tcolorbox}[title=TLDR]
    The depth of a numerical semigroup captures a lot.
\end{tcolorbox}

Let's look at the nonzero elements of a numerical semigroup \(S\).

Smallest thing in it: multiplicity \(m(S)\)
Largest thing not in it: Frobenius number \(F(S)\)

In other words, everything before \(m(S)\) is not in it (except 0).
everything after \(F(S)\) is in it. (show duality in picture with infinite line excluding 0.

So fixing \(F(S)\) and \(m(S)\) restricts a lot. What really matters is relative size, we call this the depth.

\begin{definition}
    The depth of a numerical semigroup is the unique integer \(d\) satisfying \[(d - 1) \cdot m(S) < F(S) < d \cdot m(S).\] Equivalently, \(d = \lceil \frac{F(S) + 1}{m(S)} \rceil\).
\end{definition}

\begin{tcolorbox}[title=Trivia!]
    Characterize all numerical semigroups with \(\depth(S) = 1\).
\end{tcolorbox}

What about depth 2?

\section{Depth 2}

I like to use the partition picture. Let's recall what makes a partition the one associated to a numerical semigroup.

Hookset = gapset. So we need every hook to appear in the first column, then we are safe!



\end{document}
