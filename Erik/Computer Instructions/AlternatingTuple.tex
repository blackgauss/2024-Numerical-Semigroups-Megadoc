\documentclass[11pt]{article}
\usepackage{fullpage}
\usepackage[top=2cm, bottom=4.5cm, left=2.5cm, right=2.5cm]{geometry}
\usepackage{amsmath, amsthm, amsfonts, amssymb, amscd}
\usepackage{lastpage}
\usepackage{enumerate}
\usepackage{fancyhdr}
\usepackage{mathrsfs}
\usepackage{xcolor}
\usepackage{graphicx}
\usepackage{listings}
\usepackage{hyperref}
\usepackage[capitalize, nameinlink, noabbrev]{cleveref}
\usepackage{tikz-cd}
\usepackage{tikz}
\usepackage{pgfplots}
\usepackage{todonotes}
\usepackage{physics}
\usepackage{draftwatermark}
\usepackage{ytableau}
\usepackage[breakable, skins, most]{tcolorbox}
\usepackage{Erik/Styles/colorboxes}
\usepackage{natbib}
\usepackage{url}

% Bibliography setup
\bibliographystyle{plainnat} % Specify bibliography style

\pgfplotsset{compat=1.12}

\lstdefinestyle{Sage}{
    language        = Python,
    frame           = lines,
    basicstyle      = \ttfamily\footnotesize,
    keywordstyle    = \color{blue},
    stringstyle     = \color{green},
    commentstyle    = \color{red}\ttfamily,
    showstringspaces= false,
    columns         = flexible,
    keepspaces      = true,
    breaklines      = true
}

\setlength{\parindent}{0in}
\setlength{\parskip}{0.05in}

% Edit these as appropriate
\newcommand\course{UCI 2024}
\newcommand\hwnumber{Ap\'ery Weights over Kunz Polyhedra}
\newcommand\Name{Erik Imathiu-Jones}

% Macros
\usepackage{Erik/Styles/macros}

\SetWatermarkText{} % Customize watermark text here
\SetWatermarkScale{1}
\SetWatermarkAngle{0}

\pagestyle{fancyplain}
\headheight 35pt
\lhead{\Name}
\chead{\textbf{\Large \hwnumber}}
\rhead{\course}
\lfoot{}
\cfoot{}
\rfoot{\small\thepage}
\headsep 1.5em

\title{}
\author{Erik Imathiu-Jones}
\setcounter{tocdepth}{3}
\setcounter{secnumdepth}{3}

\begin{document}

Consider the Kunz tuple \((d, 1, \dots, 1, d)\) for some \(d > 0\) and the \(d\)s and \(1\)s are alternating. Let \(n_1\) be the number of \(1\)s and \(n_d\) the number of \(d\)s.

We know \[\begin{cases} n_1 + d \cdot n_d = g \\ n_1 + n_d = m - 1\end{cases}\]

We know \(n_1 + 1 = n_d\) so \(n_d - 1 + d n_d = g\) and \((d + 1)n_d = g + 1\) so \(n_d = \frac{g+1}{d+1}\).

\iffalse
From the second equation we find \(n_1 = (m-1) - n_d\). Substituting this into the first equation yields \([(m-1) - n_d] + d \cdot n_d = g\) from which we conclude \[n_d = \frac{g - (m-1)}{d-1}.\]
\fi 

Now consider some \(1\) in the tuple. Let \(j\) be the number of \(d\)s appearing after the \(1\) so that \(n_d - j\) is the number of \(d\)s appearing before the \(1\). Every \(d\) appearing after the \(1\) contributes \(d-1\) gaps larger than the \(1\) and the ones before the \(1\) contribute \(d-2\) gaps larger than the \(1\). So the Ap\'ery weight can be expressed as 

\begin{align*}
    \mathsf{ApW}(S) &= \sum\limits_{j=0}^{n_d} (d-1)(n_d - j) + (d-2)j \\
    &= \frac{1}{2} (-3 + 2d) n_d (1 + n_d) \\
\end{align*}

Now we can substitute our expression for \(n_d\) to yield 

\iffalse\[\mathsf{ApW}(S) = \frac{(2d - 3)(d + g - m)(g - m + 1)}{2(d - 1)^2}\]

Maximize with respect to \(m\) yields \(\frac{(2d - 3)(g + 1)(d + g)}{2(d - 1)^2}
\)
\fi

\[\mathsf{ApW}(S) =  \frac{g^2(2d - 3) + g(2d^2 + 3d - 9) + 2d^2 + d - 6}{2(d+1)^2}\]
Leads to \(x = x_1 = \dots x_{d-1}\) and \(y = x_d\). With this

\begin{align*}
    \sum\limits{j=1}^d j x_j &= g \\
    \sum\limits_{j=1}^{d-1} j x  + d y &= g \\
    \frac{x}{2}(d-1)d + dy &= g\\
    \implies \frac{g}{d} - \frac{x}{2d}(d-1)d &= y  
\end{align*}
\begin{align*}
    F_1(x_1, \dots, x_{d}) &= (x_1 + 1)\sum\limits_{j=1}^{d} (j - 1) x_j \\
    &= (x + 1)\sum\limits_{j=1}^{d-1} (j - 1) x + (x + 1)(d - 1)y \\
    &= x(x+1) \cdot \frac{1}{2}(d^2 - 3d + 2) + (x+1)(d - 1)y
\end{align*}

\begin{align*}
    F_2(x_1, \dots, x_{d}) &= \sum\limits_{i = 2}^d\limits\sum\limits_{j=i+1}^d (j - i) x_i x_j \\
    &= \sum\limits_{i = 2}^{d-1}\limits\sum\limits_{j=i+1}^d (j - i) x_i x_j \\
    &= \sum\limits_{i = 2}^{d-1}\limits\sum\limits_{j=i+1}^d (j - i) x x_j \\
    &= \sum\limits_{i = 2}^{d-1}\left( (d-i)xy + \sum\limits_{j=i+1}^{d-1}(j - i) x^2\right)\\
    &= \sum\limits_{i = 2}^{d-1}\left( (d-i)xy + \frac{x^2}{2}(d-i-1)(d-i)\right)
    &= \frac{1}{6}(d^2 - 3d + 2) \cdot x \left( ((d - 3)x + 3y) \right)
\end{align*}

\begin{align*}
    F(x, y) &= x(x+1) \cdot \frac{1}{2}(d^2 - 3d + 2) + (x+1)(d - 1)y +  \frac{1}{6}(d^2 - 3d + 2) \cdot x \left( ((d - 3)x + 3y) \right) \\
    &= \frac{1}{6}(d - 1)\left(d^2x^2 - 2dx^2 + 3dxy + 3dx - 6x + 6y\right)
\end{align*}

Use \(\frac{g}{d} - \frac{x}{2d}(d-1)d = y\) to get

\[F(x) = \frac{(d - 1) \left( dx \left(-2d^2 x + 3dx(d - 1) + 4dx - 6g + 6 \right) - 12g \right)}{12d}
\]
This has critical value \[F(x_{\text{crit}}) = \frac{4d^2g + 3dg^2 - 6dg + 3d - 3g^2 + 2g - 3}{4d(d + 1)}
\] which is equal to \[\frac{3(d - 1)}{4d(d+1)}\left(g^2 + \frac{(4d - 2)}{3}g + 1\right).
\]

Want to solve
\end{document}