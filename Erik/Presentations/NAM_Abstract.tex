\documentclass[11pt]{article}
\usepackage{fullpage}

\begin{document}

\section*{Bounding The Effective Weight of Numerical Semigroups\\
Erik Imathiu-Jones}

Numerical semigroups are subsets of the natural numbers that are closed under addition and have a finite complement. The size of this complement is known as the genus of the semigroup. These structures arise in algebraic geometry through the study of Weierstrass semigroups associated with points on algebraic curves. Each numerical semigroup is generated by a finite set of natural numbers, where every element of the semigroup can be expressed as a linear combination of the generators with non-negative integer coefficients. Nathan Pflueger introduced the concept of effective weight as a numerical invariant that measures the complexity of a semigroup by counting pairs of generators and gaps, where gaps are the elements not included in the semigroup. Pflueger conjectured that, for any numerical semigroup of genus \( g \), the effective weight is bounded above by \( \left\lfloor \frac{(g + 1)^2}{8} \right\rfloor \). We present our progress toward a proof of Pflueger's conjecture. We organize numerical semigroups by associating an integer called the depth, which reflects the structure of their gaps, and have proven the conjecture for semigroups of depth 2. We also introduce a new tool called the Apéry weight, which we have used to make progress on larger depths.

\newpage

A numerical semigroup \( S \) is a subset of the natural numbers that is closed under addition and has finite complement. The elements of the complement are called the gaps of \( S \), and the number of gaps is the genus of \( S \). Numerical semigroups arise in algebraic geometry through the study of Weierstrass semigroups of algebraic curves. Each numerical semigroup \( S \) is generated by a finite set of positive integers, that is, every element of \( S \) is a linear combination of these generators with non-negative integer coefficients. Motivated by the connection to algebraic curves, Pflueger introduced the effective weight of a semigroup \( S \), which is defined in terms of the generators and gaps. He conjectured that for any semigroup of genus \( g \), the effective weight is at most \( \left\lfloor \frac{(g+1)^2}{8} \right\rfloor \). We present our progress towards a proof of this conjecture. We organize numerical semigroups in terms of their depth, an integer that reflects the structure of the set of gaps, and have proven the conjecture for semigroups of depth 2. We also introduce a new invariant called the Apéry weight, which we have used to make progress for larger depths.

\end{document}
