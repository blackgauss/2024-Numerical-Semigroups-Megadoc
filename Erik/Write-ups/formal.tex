\documentclass[11pt]{article}
\usepackage{fullpage}
\usepackage[top=2cm, bottom=4.5cm, left=2.5cm, right=2.5cm]{geometry}
\usepackage{amsmath,amsthm,amsfonts,amssymb,amscd}
\usepackage{lastpage}
\usepackage{enumerate}
\usepackage{fancyhdr}
\usepackage{mathrsfs}
\usepackage{xcolor}
\usepackage{graphicx}
\usepackage{listings}
\usepackage{hyperref}
\usepackage[capitalize,nameinlink,noabbrev]{cleveref}

\usepackage{tikz-cd}
\usepackage{tikz}
\usepackage{pgfplots}

\usepackage{todonotes}
    \pgfplotsset{
        compat=1.12,
    }
\usepackage{physics}
\usepackage{draftwatermark}
\usepackage{tikz}
\usetikzlibrary{arrows.meta}
\usetikzlibrary{positioning}
\usetikzlibrary{trees}
% \usepackage{emoji}

\usepackage{pdflscape}
\usepackage{ytableau}

\newcommand{\term}[1]{\textbf{#1}}


\usepackage[breakable,skins,most]{tcolorbox}

% tcolorboxes
\usepackage{Erik/Styles/colorboxes}
 
\renewcommand\lstlistingname{Algorithm}
\renewcommand\lstlistlistingname{Algorithms}
\def\lstlistingautorefname{Alg.}

\lstdefinestyle{Python}{
    language        = Python,
    frame           = lines, 
    basicstyle      = \footnotesize,
    keywordstyle    = \color{blue},
    stringstyle     = \color{green},
    commentstyle    = \color{red}\ttfamily
}

\setlength{\parindent}{0.0in}
\setlength{\parskip}{0.05in}

% Edit these as appropriate
\newcommand\course{UCI 2024}
\newcommand\hwnumber{}                 
\newcommand\Name{Erik Imathiu-Jones}

% Macros
\usepackage{Erik/Styles/macros}

\SetWatermarkText{}%\emoji{rose}} %\emoji{smiling-face-with-horns}}
\SetWatermarkScale{1}
\SetWatermarkAngle{0}

% Bibliography setup
\usepackage{natbib} % For more flexible citations
\bibliographystyle{plainnat} % Specify bibliography style
\usepackage{url} % To include URLs in bibliography


\pagestyle{fancyplain}
\headheight 35pt
\lhead{\Name}
\chead{\textbf{\Large \hwnumber}}
\rhead{\course}
\lfoot{}
\cfoot{}
\rfoot{\small\thepage}
\headsep 1.5em

\title{Partition Land: A Collection of Ideas from Summer 2024}
\author{Erik Imathiu-Jones}
\setcounter{tocdepth}{3}
\setcounter{secnumdepth}{3}

\begin{document}
This document is a running list of observations about the correspondences between numerical semigroups, partitions, and posets. If you want a quick overview of the facts please read the summaries below. If you want to see more details then read the full sections and please provide feedback and point out any errors. Thanks!
\tableofcontents
\section{Outline / tldr}

\subsection{Occurrences of Hooks and Gap Posets}
The gap poset of a numerical semigroup \( S \) is defined by the gaps of \( S \) ordered by the relation \( \ge_S \). Pseudo-Frobenius numbers are the maximal elements of the gap poset. Hooks in the Ferrers diagram of \( S \) correspond to gaps and their relations in the gap poset. The size of the partition equals the number of relations in the gap poset, with pseudo-Frobenius numbers being hooks that appear only once.

\subsection{Effect Weight and Cover Relations}
A pair \((x, y)\) in a poset is a covering relation if \( x >_S y \) with no intermediate element \( z \) such that \( x >_S z >_S y \). \( x - y \) is a minimal generator of \( S \) if and only if \((x, y)\) is a covering relation. We show that the effective weight equals the number of cover relations in the gap poset of \( S \). Pflueger's conjecture bounds \(\text{ewt}(S)\) by \(\left\lfloor \frac{(g + 1)^2}{8} \right\rfloor\), where \( g \) is the genus of \( S \). Can this be explained in terms of the structure of gap posets?


\newpage
\section{Occurrences of Hooks and Gap Posets}

\begin{definition}
Let \( S \) be a numerical semigroup. The \term{gaps} of \( S \) are denoted by \(\{l_1, \dots, l_g\}\) and form a partially ordered set under the relation \( \ge_S \), defined by \( l_i \ge_S l_j \) whenever \( l_i - l_j \in S \). This partially ordered set is referred to as the \term{gap poset}.
\end{definition}

\begin{proposition}
The pseudo-Frobenius numbers are the maximal elements in the gap poset.
\end{proposition}

\begin{proof}
    Suppose \( l_i \) is a maximal element in the gap poset. This means there does not exist a gap \( l_j \) such that \( l_j - l_i \in S \). In other words, \( l_i + S \subseteq S \) and \( l_i \) is a pseudo-Frobenius number.
    
    Conversely, if \( l_i \) is a pseudo-Frobenius number, then \( l_i + S \subseteq S \). Suppose, for the sake of contradiction, that there exists a gap \( l_j \) such that \( l_j - l_i \in S \). Then \( l_j = l_i + s \) for some \( s \in S \), which would imply that \( l_j \) is in \( S \), contradicting the fact that \( l_j \) is a gap.
\end{proof}


We say a row corresponds to a gap \(l_j\) if \(l_j\) appears in the first column of the given row. 

\begin{proposition}
A gap \( l_i \) appears in the row corresponding to a gap \( l_j \) if and only if \( l_j \ge_S l_i \).
\end{proposition}

\begin{proof}
Suppose \( l_i \) appears in the row of a gap \( l_j \). Then by construction of the partition, \( l_i = l_j - s \) for some \( s \in S \). Therefore, \( l_j - l_i = s \), implying \( l_j \ge_S l_i \).

Conversely, if \( l_j \ge_S l_i \), then \( l_j - l_i = s \) for some \( s \in S \). Hence, \( l_i = l_j - s \) must appear as a hook in the row of \( l_j \).
\end{proof}


As a consequence of these two propositions we have a nice characterization of pseudo-Frobenius numbers.

\begin{corollary}
    The pseudo-frobenius numbers are the hooks which only appear once in the partition of \(S\).
\end{corollary}

Similarly, the number of times a hook appears counts how many gaps are cover the hook in the gap poset.

\begin{corollary}
The number of relations in the gap poset is equal to the number of boxes in the partition, which is the size of the partition.
\end{corollary}

\iffalse
\begin{proof}
    Each relation in the gap poset corresponds to a box in the Ferrers diagram of the partition. Since every box represents a unique relation where a hook \( h \) can be determined from a gap \( f \) and a semigroup element \( s \), the total number of such relations is precisely the number of boxes in the partition.
\end{proof}
\fi

\newpage
\section{Effect Weight and Cover Relations}

Recall that a pair \((x, y)\) in a poset is a \textbf{covering relation} if \(x >_S y\) but there is no \(z\) such that \(x  >_S z\) and \(z >_S y\). We have a nice interpretation of covering relations as corresponding to minimal generators.

\begin{proposition}
    \((x, y)\) is a covering relation if and only if \(x - y\) is a minimal generator of \(S\).
\end{proposition}

\begin{proof}
    \textbf{Part 1: If \(x - y\) is not a minimal generator, then \((x, y)\) is not a covering relation.}

    Assume \(x - y = s\) is not a minimal generator. Then \(s\) can be written as \(s = s_1 + s_2\) for some non-zero \(s_1, s_2 \in S\). Let \(z = y + s_1\). We have:
    \[
    z - y = s_1 \in S \quad \text{and} \quad x - z = x - (y + s_1) = s_2 \in S.
    \]
    Therefore, \(z \ge_S y\) and \(x \ge_S z\), showing that \(z\) lies between \(x\) and \(y\) in the poset. Hence, \((x, y)\) is not a covering relation.

    \textbf{Part 2: If \((x, y)\) is not a covering relation, then \(x - y\) is not a minimal generator.}

    Assume \((x, y)\) is not a covering relation. Then there exists an element \(z\) such that \(x >_S z >_S y\). Let \(x - z = s_1\) and \(z - y = s_2\). Thus:
    \[
    x - y = (x - z) + (z - y) = s_1 + s_2.
    \]
    Since \(s_1, s_2 \in S\), \(x - y\) is expressed as a sum of nonzero elements of \(S\), implying that \(x - y\) is not a minimal generator.
\end{proof}

As of now, I can barely remember the effective weight definition so I recall here for convenience. See \cite{Pflueger_2018} and \cite{kaplan2019families} for more background.

\begin{definition}
        Let \( S \) be a numerical semigroup with gap set \( \mathbb{N}_0 \setminus S = \{ l_1, \ldots, l_g \} \).
    The \term{effective weight} of \( S \) is
    \[
    \text{ewt}(S) = \sum_{l \in \mathbb{N}_0 \setminus S} \# \{ \text{minimal generators } a < l \}.
    \]
    \[
    \text{ewt}(S) = \# \{ \text{pairs } (a, b) : 0 < a < b, \, a \text{ is a generator and } b \text{ is a gap} \}.
    \]
\end{definition}

We say a cover relation of the form \((l, l')\) is cover relation below \(l\).

\begin{proposition}
\[\# \{ \text{minimal generators } a < l_i \} = \# \{\text{cover relations below \( l_i \)}\}\]
\end{proposition}

\begin{proof}
Define a map \(\varphi\) from the set \(\{\text{minimal generators } a < l_i\}\) to the set \(\{\text{cover relations } (l_i, l_j)\}\) by:
\[
\varphi(a) = (l_i, l_i - a).
\]

First, we verify that \(\varphi\) is injective. Suppose \(\varphi(a_1) = \varphi(a_2)\) for some minimal generators \(a_1\) and \(a_2\). This means:
\[
(l_i, l_i - a_1) = (l_i, l_i - a_2).
\]
Therefore, \(l_i - a_1 = l_i - a_2\), which implies \(a_1 = a_2\).

Next, we check that \(\varphi\) is surjective. Let \((l_i, l_j)\) be a cover relation below \( l_i \). By the previous proposition, we have \(l_i - l_j = a\) where \(a\) is a minimal generator. Clearly, \(a < l_i\) since \(l_j > 0\). Thus, \(\varphi(a) = (l_i, l_j)\) with \(a = l_i - l_j\).
\end{proof}

\begin{corollary}
The effective weight of \(S\) is equal to the number of cover relations in the gap poset of \(S\).
\end{corollary}

For convenience, here is Pleuger's conjecture on the effective weight.

\begin{conjecture}[\cite{Pflueger_2018}]
    Let \( S \) be a semigroup with \( g(S) = g \). Then
    \[
    \text{ewt}(S) \leq \left\lfloor \frac{(g + 1)^2}{8} \right\rfloor.
    \]
\end{conjecture}

If the correspondences above are valid then this becomes a statement about the number of edges possible in a gap poset with \(g\) vertices.

\section{Gap Poset Dictionary}

Rec
\begin{proposition}
    The depth of \(S\) is equal to the longest chain in \((G(S), \ge_S)\).
\end{proposition}
\begin{proof}
    For brevity's sake let \(d\) denote the depth of \(S\).
    First consider the chain \(F(S) - m(S), F(S) - m(S), \dots, F(S) - (d-1) m(S)\) which has length \(d\). 

    We claim this is the longest chain of \((G(S), \ge_S)\)
\end{proof}
\bibliography{Erik/bibtex/references}

\end{document}