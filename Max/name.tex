\documentclass[11pt]{article}
\usepackage{amsmath,amsthm,amsfonts,amssymb,amscd}
\usepackage{cancel}
\usepackage[shortlabels]{enumitem}
\usepackage{fancyhdr} 
\usepackage{fullpage}
\usepackage[top=2cm, bottom=4.5cm, left=2.5cm, right=2.5cm]{geometry}
\usepackage{graphicx}
\usepackage{hyperref}
\usepackage{mathtools}
\usepackage{multicol}
\usepackage{pgfplots}
\usepackage{tikz-cd,tikz}
\usepackage{todonotes}
    \pgfplotsset{
        compat=1.12,
    }
\usepackage[breakable,skins,most]{tcolorbox}
\usepackage{verbatim}
\usepackage{xcolor}

\newtcolorbox{greybox}[1]{%untitled
colback=gray!5!white,
colframe=gray!75!black,
fonttitle=\bfseries,
title=#1}

\newtcolorbox{defbox}{
colback=blue!5,
colframe=blue!50,
fonttitle=\bfseries,
title=Definition}

\newtcolorbox{exbox}{
colback=green!5,
colframe=green!75!black,
fonttitle=\bfseries,
title=Example}

\newtcolorbox{pbox}[1]{
colback=violet!5,
colframe=violet!50,
fonttitle=\bfseries,
title=Proposition #1}

\newtcolorbox{quesbox}{
colback=magenta!5,
colframe=magenta!50,
fonttitle=\bfseries,
title=??}

\newtcolorbox{qbox}{
colback=red!5,
colframe=red!75,
fonttitle=\bfseries,
title=Question}

\newtcolorbox{thbox}[1]{
colback=violet!5,
colframe=violet!75,
fonttitle=\bfseries,
title=Theorem #1}

\hypersetup{%
  colorlinks=true,
  linkcolor=blue,
  linkbordercolor={0 0 1}
}
 
\setlength{\parindent}{0.0in}
\setlength{\parskip}{0.05in}

% Edit these as appropriate
\newcommand\course{}             
\newcommand\Name{}

% Macros
\newcommand{\A}{\mathbb{A}}
\newcommand{\C}{\mathbb{C}}
\newcommand{\F}{\mathbb{F}}
\newcommand{\calH}{\mathcal{H}}
\newcommand{\N}{\mathbb{N}}
\newcommand{\calP}{\mathcal{P}}
\newcommand{\Q}{\mathbb{Q}}
\newcommand{\R}{\mathbb{R}}
\newcommand{\Z}{\mathbb{Z}}

\newcommand{\all}{\ \forall \ }
\newcommand{\ans}{\textbf{A: \ }}
\newcommand{\aut}{\text{Aut}}
\newcommand{\blue}[1]{\textcolor{blue}{#1}}
\newcommand{\bs}{\backslash}
\newcommand{\cross}{\times}
\newcommand{\Def}[1]{\begin{defbox}{#1}
\end{defbox}}
\newcommand{\defn}{\textbf{\underline{Definition}}}
\newcommand{\ex}[1]{\begin{exbox}{#1}
\end{exbox}}
\newcommand{\exist}{ \ \exists \ }
\newcommand{\ext}{\text{Ext}}
\newcommand{\facts}{\textbf{\underline{Facts}}}
\newcommand{\green}[1]{\textcolor{green}{#1}}
\newcommand{\Hom}{\text{Hom}}
\newcommand{\id}[1]{\begin{quesbox}{#1}
\end{quesbox}}
\newcommand{\iso}{\cong}
\newcommand{\kw}[1]{\textit{#1}}
\newcommand{\la}{\lambda}
\newcommand{\prop}[2]{\begin{pbox}{#1}{#2}
\end{pbox}}
\newcommand{\Ques}[1]{\begin{qbox}{#1}
\end{qbox}}
\newcommand{\red}[1]{\textcolor{red}{#1}}
\newcommand{\sect}[1]{\section*{#1}}
\newcommand{\ssect}[1]{\subsection*{#1}}
\newcommand{\thm}[2]{\begin{thbox}{#1}{#2}
\end{thbox}}

\title{}
\pagestyle{fancyplain}
\headheight 35pt
\lhead{\Name}
\chead{\textbf{\Large Numerical Semigroups}}
\rhead{\course}
\lfoot{}
\cfoot{}
\rfoot{\small\thepage}
\headsep 1.5em

\setcounter{tocdepth}{3}
\setcounter{secnumdepth}{3}

\begin{document} 

\underline{Prop} Let $S$ be a numerical semigroup. 
$$S \subseteq A(I \cup S) \iff I \subseteq M(S) \ \text{is an order ideal}$$
\underline{Pf} Suppose $I$ is an order ideal. Suppose $s \in S$. 

Show $s + (I \cup S) \subseteq I \cup S$. 

equivalently $s+y \in I \cup S$ for any $y \in I \cup S$. 

Suppose $y \in I.$ Then $F-y \in M(S).$ 

Suppose $s+y \not\in S.$ Then either $s + y \not \in M(S)$ or $s+y \in M(S) \backslash I$. If $s + y \not \in M(S)$ then $F-S - y \in S.$ But then $(F-s-y)+s = F-y \in S$, which is a contradiction. 

If $s+y \in M(S) \backslash I$. Note: $y \preceq s+y$ since $(s+y) - y \in S.$ Since $y \in I$ and $I $ is an order ideal $s+y \in I.$

Suppose $S \subseteq A(I \cup S)$. 

Suppose $x \in I $ and $\underbrace{x \preceq y}_{y - x \in S}$

Show $y \in I.$
$(y - x)+ \underbrace{x}_{I} = y \in I \cup S$ since $y-x \in A(I \cup S)$

Since $y \in M(S), y \not \in S,$ so $y \in I. \qed$


\underline{Def} A poset $(P, \preceq)$ is self-dual if $\exist$ a bijection $\phi : P \to P$ s.t. $x \preceq y \iff \phi(y) \preceq \phi(x).$

\underline{Prop} $(M(S), \preceq)$

\underline{pf} consider $\phi(x) = F-x.$ If $x \preceq y (y -x \in S)$, then $\phi (y) \preceq \phi(x) ((F-x)-(F-y) = y-x \in S.)$

\underline{Def} An order ideal $I \subseteq M(S)$ is \underline{self-dual} if $x \in I \implies \overline{x} = F-x \in I$.  

\underline{Prop} If $I \subseteq M(S)$ is a self-dual order ideal, then $A(I \cup S) = S.$ So $T=S \cup I$ is a numerical set associated to $S.$

\underline{Ex} (from slides?) $S=\langle 9, 10,11,12,13 \rangle$ 

$|\lambda(S)| = 32$ 

$F(S)=17$

$\N_0 \backslash S = \{1,2, \dots, 8, 14, 15, 16, 17 \}$

$\lambda = (9, 8, 2, 2, 2, 2, 2, 2, 2) \quad |\lambda|=31$

$H(\lambda) = \N_0 \backslash S$. 

$\lambda$ corresponds to some numerical set 

$T = S \cup I$ for some order ideal of $(M(S),\preceq)$. 

$M(S) = \{1, 16, 2, 15, 3, 14 \}$

**Insert Image 1**

Self dual order ideal: how to create one? 

\underline{Q:} How many numerical sets associated to $S$ are of the form $T= S \cup I$ where $I $ is a self-dual order ideal of $M(S)?$

\underline{Theorem 3.6}
If $I$ is a self-dual order ideal, $I \cap PF(S)$ is a union of connected components of $GPF(S)$. 

Conversely given any union of connected components of $GPF(S)$, there exists a self-dual order ideal $I$ for which those are the maximal elements. 

$GPF(S)$

-Vertices $PF(S) \backslash F(S)$

-Edges $(P,Q)$ if $P+Q - F \in S$

-Can have loops

\underline{Note} Edge between $P, Q \iff \overline{P} \preceq Q, \overline{Q} \preceq P$. 

Check: $Q-(F-P) = P + Q - F \in S \quad P - (F-Q) = P+ Q - F \in S$

We know that $S \subseteq A(I \cup S) \iff I$ is an order ideal. 

\underline{Q:} What is the condition on an order ideal $I$ s.t. $A(I \cup S ) \subseteq S$?

\underline{Theorem 3.9} (ii) for each $P \in I \cap PF(S)$ either 

(a) $2 P \not \in S$

(b) $F-P \in I$

(c) 'there is a Frobenius Triangle $(P,x,y)$ satisfied by $I$'

What is a Frobenius Triangle? 

If $P \in PF(S) \backslash F$ and $x,y \in M(S),$ then $(P,x,y)$ is a \underline{Frobenius triangle} if $P + x = \overline{y} = F -y$, $P+x+y = F$. 

$I$ satisfies $(P,x,y)$ if $P, x \in I$ and $\overline{y} \not \in I.$

\underline{Ex} $I \in \{1, 14, 16\}$

$I \cap PF(S) = \{14,16 \}$

-$P=16, \ \overline{P} = 17 - P = 1 \in I$. 

- $P = 14, \ \overline{P} = 3 \not \in I. 2P = 14 + 14 \in S$. 

$(14, 1, 2)$ we need that $y=2, \overline{y} = 17 -2 =15 \not \in I$ so $(14, 1, 2)$ is a frobenius triangle. $(14,2,1)$ is not. 

$S$ is called \underline{elementary} if $F(S) < 2M(S)$. 

$S = \{ 0 , m , \underbrace{\dots}_{m+1, \dots 2m-1}, 2m, 2m+1, \dots \}$
(s is determined by $m+1, \dots , 2m-1$). 





\end{document}