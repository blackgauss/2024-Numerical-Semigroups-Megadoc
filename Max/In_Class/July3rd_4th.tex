I am going to look at void posets as that is something I have not spent very much time on. Here is some necessary background:

\Def{Definition}{The \textbf{void} of $S$, denoted $\mathcal{M}(S)$ is the union of all the sets $\{x, F(S) -x\}$ for all $x$ where both $x$ and $F(S)-x$ are gaps. Note $\mathcal{M}(S) \subseteq \N \setminus S$.

The \textbf{void poset} of $S$, denoted $(\mathcal{M}(S), \preceq)$, is the void, $\mathcal{M}$ equipped with the partial ordering $\preceq$, where
    $$x \preceq y \text{ if } y-x \in S \text{ for all } x, y \in \mathcal{M}(S).$$
}

\Def{Definition}{$x \in \mathcal{M}(S)$ is \textbf{maximal} if $x \preceq y \implies x = y$.

$x$ is a \textbf{pseudofrobenius number} if $x \not\in S$ and $x+s \in S$ for any $s \in S \backslash  \{ 0\}$. We denote the set of pseudofrobenius numbers by $PF(S)$.}

\prop{}{ The maximal elements of $(\calM(S),\preceq)$ are the elements of $PF(S) \backslash F(S)$.}

\Def{Definition}{Let $I \subseteq M(S).$ }

Now that I have gathered all this information in one place I am going to write/find code that displays the void poset of a given numerical semigroup (with the edges labeled by difference?). This should be useful because it will allow me to easily visualize many posets and so recognize patterns faster. 

Deepesh sent code, the first in the discord code channel that has to do with this. 





